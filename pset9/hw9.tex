\documentclass{article}
\usepackage{amssymb}
\usepackage{multirow}
\usepackage{amsmath}
\usepackage{centernot}
\usepackage{scalerel}
%\usepackage{stackengine}
\usepackage{xcolor}
\usepackage{circuitikz}
\usepackage{graphicx}
\newcommand\showdiv[1]{\overline{\smash{\hstretch{.5}{)}\mkern-3.2mu\hstretch{.5}{)}}#1}}
\newcommand\ph[1]{\textcolor{white}{#1}}


\makeatletter
% we use \prefix@<level> only if it is defined
\renewcommand{\@seccntformat}[1]{%
  \ifcsname prefix@#1\endcsname
    \csname prefix@#1\endcsname
  \else
    \csname the#1\endcsname\quad
  \fi}
% define \prefix@section
\newcommand\prefix@section{}
\newcommand\prefix@subsection{}
\makeatother

\begin{document}

\title{Digital Logic Design: Homework 9}
\author{Joshua Dong}
\date{\today}
\maketitle

\section{14.12}
\subsection{a)}
\begin{tabular}{l | l | l | l}
\multirow{2}{*}{Current State} & \multicolumn{2}{|l|}{Next State} & \multirow{2}{*}{Output} \\
                               & X=0            & X=1           &                         \\
\hline
A & B & A & 0 \\ % start
B & C & A & 0 \\ % 0
C & C & D & 0 \\ % 00
D & B & A & 1 \\ % 001
\end{tabular}

\subsection{b)}
\begin{tabular}{l | l | l | l}
X & current & next & Z  \\
\hline
0 & A & B & 0 \\
0 & B & C & 0 \\
0 & C & C & 0 \\
1 & A & A & 0 \\
1 & B & A & 0 \\
1 & C & A & 1
\end{tabular}

\section{14.18}
\subsection{a)}
\begin{tabular}{l | l | l | l | l}
\multirow{2}{*}{Current State} & \multicolumn{2}{|l|}{Next State} & \multicolumn{2}{l}{Output} \\
                               & X=0            & X=1           & X=0          & X=1         \\
\hline
0  & 0  & 8  & 1 & 1 \\
3  & 0  & 8  & 0 & 0 \\
4  & 0  & 10 & 0 & 0 \\
6  & 3  & 10 & 0 & 0 \\
8  & 4  & 12 & 0 & 0 \\
10 & 4  & 13 & 0 & 0 \\
12 & 6  & 13 & 0 & 0 \\
13 & 6  & 13 & 1 & 1 \\
\end{tabular}

\subsection{b)}
\begin{tabular}{l | l | l | l}
\multirow{2}{*}{Current State} & \multicolumn{2}{|l|}{Next State} & \multirow{2}{*}{Output} \\
                               & X=0            & X=1           &                         \\
\hline
0  & 0  & 8  & 1 \\
1  & 0  & 8  & 1 \\
2  & 1  & 9  & 1 \\
3  & 1  & 9  & 0 \\
4  & 2  & 10 & 0 \\
5  & 2  & 10 & 0 \\
6  & 3  & 11 & 0 \\
7  & 3  & 11 & 0 \\
8  & 4  & 12 & 0 \\
9  & 4  & 12 & 0 \\
10 & 5  & 13 & 0 \\
11 & 5  & 13 & 0 \\
12 & 6  & 14 & 0 \\
13 & 6  & 14 & 1 \\
14 & 7  & 15 & 1 \\
15 & 7  & 15 & 1 
\end{tabular} \\
or\\
\begin{tabular}{l | l | l | l}
\multirow{2}{*}{Current State} & \multicolumn{2}{|l|}{Next State} & \multirow{2}{*}{Output} \\
                               & X=0            & X=1           &                         \\
\hline
0  & 0  & 8  & 1 \\
3  & 0  & 8  & 0 \\
4  & 0  & 10 & 0 \\
6  & 3  & 10 & 0 \\
8  & 4  & 12 & 0 \\
10 & 4  & 15 & 0 \\
12 & 6  & 15 & 0 \\
15 & 6  & 15 & 1 
\end{tabular}

\subsection{c)}
Yes, it simply outputs the validity of the last state.


%\section{15.12}
%\subsection{a)}
%\begin{tabular}{l | l | l | l | l}
%\multirow{2}{*}{Input} & \multirow{2}{*}{Current} & \multicolumn{2}{|l|}{Next} & \multirow{2}{*}{Output} \\
%                       &                          & X = 0       & X = 1      &                         \\
%\hline
%0000 & 0  & 8  & 1 & 1 \\
%0000 & 1  & 8  & 0 & 0 \\
%0000 & 2  & 10 & 0 & 0 \\
%0000 & 3  & 10 & 0 & 0 \\
%0000 & 4  & 12 & 0 & 0 \\
%0000 & 5  & 13 & 0 & 0 \\
%0000 & 6  & 13 & 0 & 0 \\
%0000 & 7  & 13 & 1 & 1 \\
%\end{tabular}

\end{document}
