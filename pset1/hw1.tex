\documentclass{article}
\usepackage{amssymb}
\usepackage{amsmath}
\usepackage{centernot}
\usepackage{scalerel}
\usepackage{stackengine}
\usepackage{xcolor}
\newcommand\showdiv[1]{\overline{\smash{\hstretch{.5}{)}\mkern-3.2mu\hstretch{.5}{)}}#1}}
\newcommand\ph[1]{\textcolor{white}{#1}}


\makeatletter
% we use \prefix@<level> only if it is defined
\renewcommand{\@seccntformat}[1]{%
  \ifcsname prefix@#1\endcsname
    \csname prefix@#1\endcsname
  \else
    \csname the#1\endcsname\quad
  \fi}
% define \prefix@section
\newcommand\prefix@section{}
\newcommand\prefix@subsection{}
\makeatother

\begin{document}

\title{Digital Logic Design: Homework 1}
\author{Joshua Dong}
\date{\today}
\maketitle

Unless otherwise specified, all numbers are decimal.

\section{1.4}
\subsection{a)}
$1457.11
\\*
= 1457 + 0.11
\\*
= (16 \cdot 91 + 1) + 0.11
\\*
= (16 \cdot (16 \cdot 5 + 11) + 1) + 0.11
\\*
= (16 \cdot (16 \cdot 5 + 11) + 1) +
    (\frac{1}{16} \cdot 1.76)
\\*
= (16 \cdot (16 \cdot 5 + 11) + 1) +
    (\frac{1}{16} \cdot (1 + \frac{1}{16} \cdot 12.16))$
\\*
Thus 1457.11$_{10}$ = 5B1.1C$_{16}$.

\section{1.7}
\subsection{b)}
$-14 = 110010_2,\; -32 = 100000_2$ in 2's complement representation.
\\*
\begin{tabular}{c@{\,}c@{\,}c@{\,}c@{\,}c@{\,}c@{\,}c}
  &1&1&0&0&1&0 \\
+ &1&0&0&0&0&0 \\
\hline
 1&0&1&0&0&1&0 \\
\end{tabular}
\\*
The result, after discarding the carry from the sign bit, is positive. It is clear that there is an overflow.

\subsection{c)}
$-25 = 100111_2,\; 18 = 010010_2$ in 2's complement representation.
\\*
\begin{tabular}{c@{\,}c@{\,}c@{\,}c@{\,}c@{\,}c@{\,}c}
  &1&0&0&1&1&1 \\
+ &0&1&0&0&1&0 \\
\hline
  &1&1&1&0&0&1 \\
\end{tabular}
\\*
There is no carry from the sign bit and there is no overflow.
\\*
Using 1's complement representation, $-25 = 100110_2,\; 18 = 010010_2$.
\\*
\begin{tabular}{c@{\,}c@{\,}c@{\,}c@{\,}c@{\,}c@{\,}c}
  &1&0&0&1&1&0 \\
+ &0&1&0&0&1&0 \\
\hline
  &1&1&1&0&0&0 \\
\end{tabular}
\\*
There is no carry from the sign bit and there is no overflow.


\section{1.8}
If 2's complement is used, the system's range is
$-2^7$ to $+(2^7 - 1)$, or $-128$ to $+127$.
If 1's complement is used, the system's range is
$-(2^7 - 1)$ to $+(2^7 - 1)$, or $-127$ to $+127$.


\section{1.10}
\subsection{a)}
$1305.375
\\*
= 1305 + 0.375
\\*
= (16 \cdot 81 + 9) + 0.375
\\*
= (16 \cdot (16 \cdot 5 + 1) + 9) + 0.375
\\*
= (16 \cdot (16 \cdot 5 + 1) + 9) +
    \frac{1}{16} \cdot 6$
\\*
Thus 1305.375$_{10}$ = 519.6$_{16}$.
\\*
Conversion to decimal is easy by converting each hexidecimal
digit to its binary representation:
\\*
0101 0001 1001 . 0110
\\*
That is,
519.6$_{16}$ = 010100011001.0110$_2$.

\subsection{d)}
$1644.875
\\*
= 1644 + 0.875
\\*
= (16 \cdot 102 + 12) + 0.875
\\*
= (16 \cdot (16 \cdot 6 + 6) + 12) + 0.875
\\*
= (16 \cdot (16 \cdot 6 + 6) + 12) +
    \frac{1}{16} \cdot 14$
\\*
Thus 1644.875$_{10}$ = 66C.E$_{16}$.
\\*
Conversion to decimal is easy by converting each hexidecimal
digit to its binary representation:
\\*
0110 0110 1100 . 1110
\\*
That is,
66C.E$_{16}$ = 011001101100.1110$_2$.


\section{1.19}
\subsection{b)}
Shown below is division with binary numbers:
\\*
\setstackgap{S}{1.5pt}
\stackMath\def\stackalignment{r}
\(
\stackunder{%
    1110 \stackon[1pt]{\showdiv{110000001}}{11011}%
}{%
    \Shortstack[l]{
        {\underline{1110}}
            \ph{}10100 {
            \ph{1}\underline{1110}}
                \ph{11}11000 {
                \ph{111}\underline{1110}} %
                    \ph{111}10101 {
                    \ph{1111}\underline{1110}}
                        \ph{11111}111}%
}
\)
\\*
\\*
$14 \cdot 27 + 7 = 385$,
\\*
$1110_2 \cdot 11011_2 + 111_2 = 110000001_2$,
\\*
The answer of $11011_2$ is correct, with remainder $111_2$.


\section{1.37}
\subsection{a)}
\begin{tabular}{c@{\,}c@{\,}c@{\,}c@{\,}c@{\,}c}
  &0&1&0&0&1 \\
- &1&1&0&1&0 \\
\hline
\end{tabular}
\\*\\*
We interpret this as 1's complement:
\\*
\begin{tabular}{c@{\,}c@{\,}c@{\,}c@{\,}c@{\,}c}
  &0&1&0&0&1 \\
+ &0&0&1&0&1 \\
\hline
  &0&1&1&1&0 \\
\end{tabular}
\\*\\*
And as 2's complement:
\\*
\begin{tabular}{c@{\,}c@{\,}c@{\,}c@{\,}c@{\,}c}
  &0&1&0&0&1 \\
+ &0&0&1&1&0 \\
\hline
  &0&1&1&1&1 \\
\end{tabular}
\\*\\*
No carry occurs on the sign bits, so there is no overflow.

\subsection{c)}
\begin{tabular}{c@{\,}c@{\,}c@{\,}c@{\,}c@{\,}c}
  &1&0&1&1&0 \\
- &0&1&1&0&1 \\
\hline
\end{tabular}
\\*\\*
We interpret this as 1's complement:
\\*
\begin{tabular}{c@{\,}c@{\,}c@{\,}c@{\,}c@{\,}c}
  &1&0&1&1&0 \\
+ &1&0&0&1&0 \\
\hline
 1&0&1&0&0&0 \\
\hline
  &0&1&0&0&1 \\
\end{tabular}
\\*\\*
And as 2's complement:
\\*
\begin{tabular}{c@{\,}c@{\,}c@{\,}c@{\,}c@{\,}c}
  &1&0&1&1&0 \\
+ &1&0&0&1&1 \\
\hline
 1&0&1&0&0&1 \\
\hline
  &0&1&0&0&1 \\
\end{tabular}
\\*\\*
There is overflow with both methods.

\subsection{e)}
\begin{tabular}{c@{\,}c@{\,}c@{\,}c@{\,}c@{\,}c}
  &1&1&1&0&0 \\
- &1&0&1&0&1 \\
\hline
\end{tabular}
\\*\\*
We interpret this as 1's complement:
\\*
\begin{tabular}{c@{\,}c@{\,}c@{\,}c@{\,}c@{\,}c}
  &1&1&1&0&0 \\
+ &0&1&0&1&0 \\
\hline
 1&0&0&1&1&0 \\
\hline
  &0&0&1&1&1 \\ 
\end{tabular}
\\*\\*
And as 2's complement:
\\*
\begin{tabular}{c@{\,}c@{\,}c@{\,}c@{\,}c@{\,}c}
  &1&1&1&0&0 \\
+ &0&1&0&1&1 \\
\hline
 1&0&0&1&1&1 \\
\hline
  &0&0&1&1&1 \\
\end{tabular}
\\*\\*
There is no overflow.


\section{2.3}
\subsection{a)}
$X'Y'Z + (X'Y'Z)' = 1X'Y'Z + 1(X'Y'Z)' = 1$, by the uniting theorem of boolean addition.
\subsection{f)}
$(A + BC) + (DE + F)(A + BC)' = (A + BC) + (DE + F)$, by the elimination theorem of boolean addition.


\section{2.5}
\subsection{a)}
$(A + B)(C + B)(D' + B)(ACD' + E)
\\*
= (A(C + B) + B(C + B))(D' + B)(ACD' + E)
\\*
= (AC + AB + BC + B^2)(D' + B)(ACD' + E)
\\*
= (AC + B)(D' + B)(ACD' + E)
\\*
= (AC + B)(D'(ACD' + E) + B(ACD' + E))
\\*
= (AC + B)(AC(D')^2 + D'E + ABCD' + BE)
\\*
= (AC + B)(ACD' + BE + D'E)
\\*
= AC(ACD' + BE + D'E) + B(ACD' + BE + D'E)
\\*
= A^2C^2D' + ACBE + ACD'E + BACD' + B^2E + BD'E)
\\*
= ACD' + ACBE + ACD'E + BACD' + BE + BD'E)
\\*
= ACD' + BE
$
\subsection{b)}
$(A' + B + C')(A' + C' + D)(B' + D')
\\*
= (A'(B' + D') + B(B' + D') + C'(B' + D'))(A' + C' + D)
\\*
= (A'B' + A'D' + BB' + BD' + C'B' + C'D')(A' + C' + D)
\\*
= (A'B' + A'D' + BD' + B'C' + C'D')(A' + C' + D)
\\*
= A'(A'B' + A'D' + BD' + B'C' + C'D') +
C'(A'B' + A'D' + BD' + B'C' + C'D') +
D(A'B' + A'D' + BD' + B'C' + C'D')
\\*
= A'A'B' + A'A'D' + A'BD' + A'B'C' + A'C'D' +
C'A'B' + C'A'D' + C'BD' + C'B'C' + C'C'D' +
DA'B' + DA'D' + DBD' + DB'C' + DC'D'
\\*
= A'B' + A'D' + A'BD' + A'B'C' + A'C'D' +
A'B'C' + A'C'D' + BC'D' + B'C' + C'D' +
A'B'D + 0 + 0 + B'C'D + 0
\\*
= A'B' + A'D' + B'C' + C'D'
$

\section{2.6}
\subsection{a)}
$AB + C'D'
\\*
= (AB + C')(AB + D')
\\*
= (A + C')(B + C')(A + D')(B + D')
$
\subsection{f)}
$A + BC + DE
\\*
= (A + B)(A + C) + DE
\\*
= ((A + B)(A + C) + D)((A + B)(A + C) + E)
\\*
= ((A + B + D)(A + C + D))((A + B + E)(A + C + E))
\\*
= (A + B + D)(A + C + D)(A + B + E)(A + C + E)
$


\section{2.8}
\subsection{a)}
$[(AB)' + C'D]' = AB(C'D)' = AB(C + D') = ABC + ABD'
$


\section{2.9}
\subsection{a)}
$F = ((A + (A + B)')' + (A + B)')(A + (A + B)')'
\\*
= ((A'(A + B)) + (A'B'))(A + A'B')'
\\*
= ((A'(A + B)) + (A'B'))(A' + (A'B')')
\\*
= ((A'(A + B)) + (A'B'))(A' + A + B)
\\*
= ((A'(A + B)) + (A'B'))(1)
\\*
= (A'A + A'B + (A'B'))
\\*
= 0 + A'B + A'B'
\\*
= A'
$
\subsection{b)}
$G = (((R + S + T)'P((R + S)'T))'T)'
\\*
= (R + S + T)'P((R + S)'T) + T'
\\*
= R'S'T'PR'S'T + T'
\\*
= PR'S'TT' + T'
\\*
= 0 + T'
\\*
= T'
$


\section{2.12}
\subsection{a)}
$(X + Y'Z) + (X + Y'Z)' = 1(X + Y'Z) + 1(X + Y'Z)' = 1$,
by the unification theorem of boolean addition.
\subsection{f)}
$(V' + U + W) [(W + X) + Y + UZ'] + [(W + X) + UZ' + Y]
\\*
= (V' + U + W) [(W + X) + UZ' + Y] + [(W + X) + UZ' + Y]
\\*
= V' + U + W
$,
by the absorption theorem of boolean addition.

\section{2.15}
\subsection{a)}
$f(A, B, C, D) = [A + (BCD)'][(AD)' + B(C' + A)]$
\\*
$dual(f)
= [A(B + C + D)'] + [(A + D)'(B + C'A)]$
\\*
$complement(f)
= [A'(B' + C' + D')'] + [(A' + D')'(B' + CA')]
\\*
= A'BCD + AD(B' + CA')$
\\*


\section{2.16}
\subsection{a)}
$f(A, B, C, D) = [A + (BCD)'][(AD)' + B(C' + A)]$
\\*
$dual(f)
= [A(B + C + D)'] + [(A + D)'(B + C'A)]$

\section{2.21}
$
\begin{array}{ccc|cccc}
A&B&C&&F&G&H
\\\hline
1&1&1&&1&\mathbf{X}&1\\
1&1&0&&0&\mathbf{0}&0\\
1&0&1&&0&\mathbf{1}&1\\
1&0&0&&0&\mathbf{0}&0\\
0&1&1&&1&\mathbf{X}&1\\
0&1&0&&0&\mathbf{1}&1\\
0&0&1&&1&\mathbf{X}&1\\
0&0&0&&0&\mathbf{0}&0
\end{array}
$
\\*
X represents an unspecified value in the above table.


\section{2.22}
\subsection{a)}
$A'B' + A'CD + A'DE'
\\*
= A'(B' + CD + DE')
\\*
= A'((B' + C)(B' + D) + DE')
\\*
= A'((B' + C + DE')(B' + D + DE'))
\\*
= A'(B' + C + DE')(B' + D)
\\*
= A'(B' + C + D)(B' + C + E')(B' + D)
\\*
= A'(B' + D)(B' + C + E')
$


\section{2.24}
\subsection{a)}
$[(XY')' + (X' + Y)'Z]
= (X' + Y) + (XY')Z
= X' + Y + XY'Z
$


\end{document}
